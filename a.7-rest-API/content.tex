% DO NOT COMPILE THIS FILE DIRECTLY!
% This is included by the other .tex files.

\begin{frame}[t,plain]
\titlepage
\end{frame}

\begin{frame}
\frametitle{MISP API reworked}
    \begin{itemize}
        \item The MISP API has grown gradually with a UI first design in many cases
        \item Endpoints all solved specific issues with their own rulesets
	\item Growth was organic - whenever the need to add a new functionality / filter popped up we've added it
	\item Lead to frankenmonsters such as this:
		\begin{adjustbox}{width=\textwidth,height=.5cm,keepaspectratio}
		    http://localhost:5000/events/csv/download/false/false/tag1\&\&tag2\&\&\!tag3/Network\%20activity/domain
		\end{adjustbox}
    \end{itemize}
\end{frame}

\begin{frame}
\frametitle{Goals we've set for ourselves}
    \begin{itemize}
        \item Open up every functionality in MISP available via the UI to the API
	\item Including ones related to {\bf instance management}
	\item APIs that expect input objects for data creation should be {\bf self-describing}
	\item {\bf URL parameters should be discouraged}, but still usable by legacy tools (deprecation)
	\item APIs should be heavily {\bf tested} (Raphael Vinot's exhaustive test suite in PyMISP)
	\item Largest focus on Export APIs
    \end{itemize}
\end{frame}


\begin{frame}
\frametitle{Export API's reimagined}
    \begin{itemize}
        \item Scrapped all existing type specific APIs ({\bf deprecated}, documentation moved to legacy, still available)
	\item {\bf Single entry point} - all export APIs baked into restSearch
	\item Queries consist of a combination of:
	\begin{itemize}
		\item {\bf Scope} (Event, Attribute, Sighting, more coming in the future)
		\item {\bf Filter parameters} - passed via JSON objects, url parameters (key value or ordered list)
		\item A {\bf return format}
	\end{itemize}
	\item Everything that we could do before the rework we should be able to accomplish after the rework
	\item Under the hood now also used by the UI search and exports
    \end{itemize}
\end{frame}

\begin{frame}
\frametitle{Export API's reimagined}
    \begin{itemize}
	\item One of our largest issues solved: {\bf pagination}
	\begin{itemize}
		\item {\bf Scope specific} pagination (number of events, attributes, etc)
		\item Simply control it via the framework friendly {\bf page / limit} parameters
		\item Alternatively, use the improved {\bf time based controls} (timestamp, publish\_timestamp windows)
	\end{itemize}
    \end{itemize}
\end{frame}

\begin{frame}
\frametitle{Performance tuning}
    \begin{itemize}
	\item Single execution with subqueries
	\item Internal pagination {\bf aligned with memory limits}
	\begin{itemize}
		\item Probing of available memory for the current process
		\item {\bf Chunking of the query results} to fit in object specific memory envelopes
		\item Constructing export set on disk in chunks has slashed memory usage considerably
	\end{itemize}
    \end{itemize}
\end{frame}

\begin{frame}
\frametitle{Designing tools that use the APIs can be complex, but there's help}
    \begin{itemize}
	\item The result of our own frustration
	\item Built in {\bf ReST client} with templating
        \item Extensive query builder UI by Sami Mokaddem
	\item Build queries in a simple interface, automatically set URLs, headers, etc
	\item Uses the self documentation of APIs
	\item Export your queries as {\bf cURL or Python scripts}
	\item Built in testing tools (performance measurements, result parsers)
        \item Store queries for reuse and download the results directly
    \end{itemize}
\end{frame}

\begin{frame}
\frametitle{Why is the search API receiving so much focus?}
    \begin{itemize}
	\item The {\bf maturity} of the communities and threat intel sharing at large has improved
	\item We are sharing more
	\item Most importantly: we are sharing {\bf more context} along with technical indicators
	\item This allows us to {\bf manage our data more accuractely} before feeding them to our protective tools
	\item Different contexts (APT targeting me? Persisting techniques?) - lifecycle management
	\item Use several queries / boolean operators to select the slice of data most relevant for the task
    \end{itemize}
\end{frame}

\begin{frame}
\frametitle{CLI tools for the CLI God, automation for the automation throne}
    \begin{itemize}
	\item Open up commonly used system management tasks to the CLI
        \begin{itemize}
            \item sync servers/feeds
            \item caching feeds
            \item Password resets
            \item Server settings
            \item Bruteforce protection resets
            \item Enrichment
            \item Worker management
        \end{itemize}
        \item Goal was also to move away from the often malfunctioning scheduler and have cron friendly CLI scripts
    \end{itemize}
\end{frame}

\begin{frame}
\frametitle{So what does all of this look like in practice?}
        \begin{center}
        	\item Demo time!
        \end{center}
\end{frame}

\begin{frame}
\frametitle{Plans for the future}
    \begin{itemize}
	\item Add export modules to the restSearch API
        \item Improve the query language to support some missing features (such as AND boolean operators)
        \item Support for extended events via the restSearch API
	\begin{itemize}
	    \item We're missing a framing structure in the export module system (how are a list of conversions encapsulated and delimited?)
            \item Proof of concept of the system implemented by Christian Studer already works using the STIX / STIX2 export subsystems
	    \item Would open us up to simple customiseable search APIs
	\end{itemize}
	\item Open up search APIs to other scopes (objects, users, organisations, proposals, feeds, galaxies, taxonomies)
    \end{itemize}
\end{frame}

